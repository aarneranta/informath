\documentclass{report}
\usepackage{amsmath, amsfonts}

\begin{document}

Thm01. If $m \in \mathbb{N}$ and $n \in \mathbb{N}^+$ then $(m/n)^2 \neq 2$.

Thm01. There are no pairs $(m, n) \in \mathbb{N}\times\mathbb{N}$ such that $n > 0$ and $(m/n)^2 = 2$.

Thm01a. The equation $(m/(n+1))^2 = 2$ has no solution in the natural numbers.

Thm01b. The equation $q^2 = 2$ has no solution in the rationals.

Thm01b. There is no rational number whose square is $2$.

Thm01b. If $q \in \mathbb{Q}$ then $q^2 \neq 2$.

Thm01e. The square root of $2$ is irrational.

Thm01e. $\sqrt{2}$ is irrational.

Thm01f. The equation $p^2 = 2q^2$ has no solution in the rationals.

Thm01f. There are no rational numbers $p$ and $q$ such that $p^2 = 2q^2$.

Thm02. Every polynomial of degree greater than zero has at least one complex root.

Thm02. Every polynomial of non-zero degree has at least one complex root.

Thm02. Every polynomial of non-zero degree has a complex root.

Thm03. The set of rational numbers is countable.

Thm03. $\mathbb{R}$ is countable.

Thm03a. $\mathbb{R}$ and $\mathbb{N}$ have equal cardinality.

Thm04. If the vectors $u$ and $v$ are perpendicular, then $\lvert u + v \rvert = \sqrt{\lvert u \rvert^2 + \lvert v \rvert^2}$.

Thm04. If $u$ and $v$ are vectors such that $u \perp v$, then $\lvert u + v \rvert = \sqrt{\lvert u \rvert^2 + \lvert v \rvert^2}$.

Thm07. If $p$ and $q$ are prime numbers, then $\left(\frac{p}{q}\right) \left(\frac{q}{p}\right) = (-1)^{\frac{p-1}{2} \cdot \frac{q-1}{2}}$.

Thm07. Suppose that $p$ and $q$ are prime numbers. Then $\left(\frac{p}{q}\right) \left(\frac{q}{p}\right) = (-1)^{\frac{p-1}{2} \cdot \frac{q-1}{2}}$.

Thm09. Any circle of radius $r$ has area $\pi r^2$.

Thm09. If a circle has radius $r$ then its area is $\pi r^2$.

Thm09. The area of a circle is $\pi r^2$, where $r$ is the radius of the circle.

Thm10FermatLittle. Suppose that $p$ is a prime number and $a$ is an integer. Then $a^p - a = pq$ for some integer $q$.

Thm10FermatLittle. If $p$ is a prime number and $a$ is an integer, then $p \mid a^p - a$.

Thm11. For every $n \in \mathbb{N}$, there is a prime $p \geq n$.

Thm11. There is no natural number which is greater than every prime number.

Thm19. Every natural number can be written as the sum of four squares.

Thm19. Every natural number can be written in the form $a^2+b^2+c^2+d^2$ where $a, b, c, d \in \mathbb{N}$.

Thm20a. Every prime number of the form $4k+1$ can be written as the sum of two squares.

Thm20a. If $p=4k+1$ is a prime number then $p$ can be written as the sum of two squares.

Thm20b. Every prime number congruent to $1$ modulo $4$ can be written as the sum of two squares.

Thm20b. Suppose $p$ is a prime such that $p \equiv 1$ (mod 4). Then $p$ can be written as the sum of two squares.

Thm22. The set of real numbers is uncountable.

Thm22. There are uncountably many reals.

Thm22. $\mathbb{R}$ is uncountable.

Thm51wilson. A natural number $n$ is prime iff $(n-1)! \equiv -1$ (mod $n$).

Thm51b. A natural number $n$ is prime iff $n \mid (n-1)! + 1$.

Thm52. If $A$ is a finite set, then $\lvert P(A) \rvert = 2^{\lvert A \rvert}$.

Thm52. If $A$ is a finite set, then its powerset has cardinality $2^{\lvert A \rvert}$.

Thm58. There are $\binom{n}{k}$ ways to choose $k$ items from a set of $n$ items, where $0 \leq k \leq n$.

Thm58. A finite set of cardinality $n$ has $\binom{n}{k}$ subsets of cardinality $k$, where $0 \leq k \leq n$.

Thm58. Let $A$ be a finite set of cardinality $n$ and $0 \leq k \leq n$. Then $A$ has $\binom{n}{k}$ subsets of cardinality $k$.

Thm78. $u \cdot v \leq \lvert u \rvert \lvert v \rvert$.

Thm78. If $u$ and $v$ are vectors, then $u \cdot v \leq \lvert u \rvert \lvert v \rvert$.

Thm78a. If two vectors are orthogonal then their dot product is $0$.

Thm78a. The dot product of any two orthogonal vectors is $0$.

Thm78a. If $u \perp v$ then $u \cdot v = 0$.

Thm78a. Let $u$ and $v$ be vectors. If $u \perp v$ then $u \cdot v = 0$.

Thm78a. If $u$ and $v$ are orthogonal vectors then $u \cdot v = 0$.

Thm91. $\lvert u+v \rvert \leq \lvert u \rvert+\lvert v \rvert$.

Thm91. Let $u$ and $v$ be vectors. Then $\lvert u+v \rvert \leq \lvert u \rvert+\lvert v \rvert$.

Thm91. If $u$ and $v$ are vectors, then $\lvert u+v \rvert \leq \lvert u \rvert+\lvert v \rvert$.

Thm98. For all $n \in \mathbb{N}$, there exists a prime between $n$ and $2n$ exclusive.

Thm98. For all $n \in \mathbb{N}$, there exists a prime between $n+1$ and $2n-1$.

Thm98. For all $n \in \mathbb{N}$, the interval $(n, 2n)$ contains a prime.

Thm98. For all $n \in \mathbb{N}$, there exists a prime $p$ such that $n < p < 2n$.

\end{document}
